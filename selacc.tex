\documentclass[modern]{aastex62}

\usepackage{amsmath}
\usepackage{amssymb}

\newcommand{\dd}{\mathrm{d}}
\newcommand{\diff}[2]{\frac{\dd #1}{\dd #2}}

\begin{document}

\title{Accuracy Requirements for Empirically-Measured Selection Functions}

\author[0000-0003-1540-8562]{Will M. Farr}
\email{will.farr@stonybrook.edu}
\email{wfarr-vscholar@flatironinstitute.org}
\affiliation{Department of Physics and Astronomy, Stony Brook University, Stony Brook NY 11794, United States}
\affiliation{Center for Computational Astronomy, Flatiron Institute, New York NY 10010, United States}

\maketitle

When conducting a population analysis on a dataset consisting of measurements
$d_i$, $i = 1, \ldots, N$, subject to some selection function that constrain the
parameters $\theta_i$ of a set of $N$ objects to infer the population
distribution
%
\begin{equation}
  \diff{N}{\theta}\left( \lambda \right),
\end{equation}
%
which can depend on some population-level parameters $\lambda$, the joint
posterior for the object-level parameters $\theta_i$ and population-level
parameters takes the form \citep{Loredo2004,Mandel2018}
%
\begin{equation}
\pi \propto \prod_{i=1}^N \left[ p\left( d_i \mid \theta_i \right) \diff{N}{\theta_i}\left( \lambda \right) \right] \exp\left[ - \Lambda\left( \lambda \right) \right].
\end{equation}
%
Here $p\left( d \mid \theta\right)$ is the likelihood function that describes
the measurement process of each system and $\Lambda$ is the expected number of
detections given the measurement process and selection effects, here assumed to
depend only on the observed data for each object:
\begin{equation}
  \Lambda\left( \lambda \right) \equiv \int_{\left\{ d \mid f(d) > 0 \right\}} \dd d \, \dd \theta \, \diff{N}{\theta}\left( \lambda \right) p\left( d \mid \theta \right),
\end{equation}
where $f$ is the selection function, such that an object will be included in the observed sample if and only if it generates data such that $f(d) > 0$.

\bibliography{selacc}

\end{document}
